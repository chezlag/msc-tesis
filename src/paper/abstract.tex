\textbf{Abstract:} This thesis studies the indirect effects of the change from paper to digital vouchers (e-invoicing) on Value Added Tax (VAT) compliance in Uruguayan firms. E-invoicing creates a record of sales for businesses that issue vouchers and a record of purchases for businesses that receive them. Therefore, the policy can help the tax authority to reduce sales under-reporting by e-invoice emitting firms and also cost over-reporting by e-invoice recepient firms. The latter indirect effect of e-invoicing has not received sufficient attention, which is an important constraint for a comprehensive evaluation of the policy.

I combine administrative records of VAT returns (2009-2016) with data from the e-invoicing system to study the first years of the implementation of the policy. I use a balanced panel of small firms and apply a difference-in-differences methodology, exploiting the variation in the dates when firms receive their first e-invoice to estimate the indirect impact of the policy. 

I analyse the effect of e-invoice receipt on input VAT, output VAT and net VAT liability. I find no statistically significant effects on any of the outcomes analysed. At the extensive margin, I find a 1-2\% increase in the probability that recipient firms start reporting input VAT and output VAT. Several robustness exercises are conducted which confirm the stability of the results.

An exploration of the e-invoicing system data shows that in the analysis period there were a very small number of businesses issuing vouchers and that the percentage of purchases by recipient businesses that are recorded in e-invoices is small. At the end of the period, for 75\% of the recipient firms, purchases captured in e-invoices represented less than half of the purchases later reported. In this context, spillovers are expected to be close to zero, as the policy would not have generated enough information on recipient firms to be able to verify inconsistencies in reported costs.

With this in mind, the absence of an effect should be understood as a highly local, partial and short-term result. The implementation of e-invoicing in Uruguay reached all firms only in 2023, and evaluations of similar policies in other countries suggest that the medium- and long-term effects may be substantially larger (Bellon et~al. 2022; Fan et al.~2023). In the Uruguayan case, more recent data on firm activity is needed to assess e-invoicing comprehensively.
