\textbf{Resumen:} Esta tesis estudia los efectos indirectos del cambio de comprobantes en papel a comprobantes digitales (e-facturación) sobre el cumplimiento tributario del Impuesto al Valor Agregado (IVA) en empresas uruguayas. La e-facturación crea un registro de ventas de las empresas que emiten comprobantes y un registro de las compras de las empresas que los reciben. Por lo tanto, la política puede ayudar a la autoridad tributaria a reducir el sub-reporte de ventas de las empresas vendedoras y también el sobre-reporte de costos de las empresas compradoras. Este segundo efecto indirecto de la e-facturación no ha recibido suficiente atención, lo cual es una limitante importante para hacer una evaluación integral de la política.

Combino registros administrativos de declaraciones juradas de IVA (2009–2016) con datos provenientes del sistema de e-facturación para estudiar los primeros años de la implementación de la política. Utilizo un panel balanceado de empresas pequeñas y aplico una metodología de diferencias en diferencias, explotando la variación en las fechas en que las empresas reciben su primera e-factura para estimar el impacto indirecto de la política. 

Analizo el efecto de la recepción de e-facturas sobre IVA compras, IVA ventas e IVA adeudado. No encuentro efectos estadísticamente significativos sobre ninguna de las variables analizadas. En el margen extensivo, encuentro un aumento de 1–2\% en la probabilidad de que las empresas receptoras empiecen a declarar IVA compras e IVA ventas. Se realizan varios ejercicios de robustez que confirman la estabilidad de los resultados. 

Una exploración de los datos del sistema de e-facturación muestra que en el período de análisis había un número muy reducido de empresas emitiendo comprobantes y que el porcentaje de compras de las empresas receptoras que queda registrado en e-facturas es acotado. Al final del período, para el 75\% de las empresas receptoras las compras capturadas en e-facturas representaban menos de la mitad de las compras luego declaradas. En este contexto, es esperable que los efectos indirectos sean cercanos a cero, en tanto la política no habría generado suficiente información sobre las empresas receptoras como para poder verificar inconsistencias en los costos reportados.

Con esto en mente, la ausencia de efecto debería entenderse como un resultado altamente local, parcial y de corto plazo. La implementación de la e-facturación en Uruguay alcanzó a todas las empresas recién en 2023, y evaluaciones de políticas similares en otros países sugieren que los efectos en el mediano y largo plazo pueden ser sustancialmente mayores (Bellon et~al. 2022; Fan et~al. 2023) En el caso uruguayo, es necesario contar con datos más recientes de la actividad de las empresas para evaluar la e-facturación de forma integral.
